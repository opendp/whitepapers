\documentclass[11pt,a4paper]{article}
\usepackage{amsmath,amsthm,amsfonts,amssymb,amscd}
\usepackage{enumerate} 
\usepackage{physics}
\usepackage{enumerate}
\usepackage{fancyhdr}
\usepackage{hyperref}
\usepackage{graphicx}
\hypersetup{colorlinks,
    linkcolor=blue,
    citecolor=blue,      
    urlcolor=blue,
    %linktoc=none
}

\oddsidemargin0.1cm 
\evensidemargin0.8cm
\textheight22.7cm 
\textwidth15cm \topmargin-0.5cm

\newtheorem{theorem}{Theorem}[section]
\newtheorem{corollary}{Corollary}
\newtheorem{lemma}{Lemma}[section]
\newtheorem{proposition}{Proposition}
\newtheorem*{theorem-non}{Theorem}

\theoremstyle{definition}
\newtheorem{remark}{Remark}
\newtheorem{definition}{Definition}[section]
\newtheorem{observation}{Observation}
\newtheorem{note}{Note}
\newtheorem{hope}{Hope}
\newtheorem{warning}{Warning}
\newtheorem{problem}{Problem}
\newtheorem{fear}{Fear}
\newtheorem{question}{Question}
\newtheorem{example}{Example}
\newtheorem{claim}{Claim}

\newcommand{\Z}{\mathbb{Z}}
\newcommand{\R}{\mathbb{R}}
\newcommand{\C}{\mathbb{C}}
\newcommand{\Q}{\mathbb{Q}}
\newcommand{\A}{\mathbb{A}}
\newcommand{\horizline}{\noindent\rule{\textwidth}{1pt}}

\usepackage{listings}
\usepackage{xcolor}

\definecolor{codegreen}{rgb}{0,0.6,0}
\definecolor{codegray}{rgb}{0.5,0.5,0.5}
\definecolor{codepurple}{rgb}{0.58,0,0.82}
\definecolor{backcolour}{rgb}{0.95,0.95,0.92}

\lstdefinestyle{mystyle}{
    backgroundcolor=\color{backcolour},   
    commentstyle=\color{codegreen},
    keywordstyle=\color{magenta},
    numberstyle=\tiny\color{codegray},
    stringstyle=\color{codepurple},
    basicstyle=\ttfamily\footnotesize,
    breakatwhitespace=false,         
    breaklines=true,                 
    captionpos=b,                    
    keepspaces=true,                 
    numbers=left,                    
    numbersep=5pt,                  
    showspaces=false,                
    showstringspaces=false,
    showtabs=false,                  
    tabsize=2
}

\lstset{style=mystyle}

\newcommand{\MultiSet}{\mathrm{MultiSets}}
\newcommand{\len}{\mathrm{len}}
\newcommand{\din}{d_{in}}
\newcommand{\dout}{d_{out}}
\newcommand{\Relation}{\mathrm{Relation}}
\newcommand{\X}{\mathcal{X}}
\newcommand{\Y}{\mathcal{Y}}
\newcommand{\True}{\texttt{True}}
\newcommand{\False}{\texttt{False}}
\newcommand{\clamp}{\texttt{clamp}}
\newcommand{\questionc}[1]{\textcolor{red}{\textbf{Question:} #1}}
\newcommand{\T}{\texttt{T}}

\newcommand{\silvia}[1]{{ {\color{blue}{(silvia)~#1}}}}
\newcommand{\grace}[1]{{ {\color{purple}{(grace)~#1}}}}
\newcommand{\connor}[1]{{ {\color{teal}{(connor)~#1}}}}

\newcommand{\todo}{{\textcolor{red}{TODO }}}

\title{List of definitions used in the proofs}
\author{S\'ilvia Casacuberta, Grace Tian, and Connor Wagaman}
\date{June--July 2021}

\begin{document}

\maketitle

We use the following guideline: if a term appears in the preconditions \& pseudocode section, then this term is defined in the ``List of definitions used in the pseudocode" document. Otherwise, it appears in the ``List of definitions used in the proofs" document. 

In both cases, we maintain the terms in alphabetical order within each section. `TODOs'' should be included at the end of the corresponding section. On the other hand, ``TODOs'' which better specify an already-defined term should be included immediately following the definition of that term. Examples should never be part of the definition, but we encourage their use right after the definition of a term.

\tableofcontents

\section*{List of terms that have not yet been added}
\begin{itemize}
\item Generalization of the path property and the corresponding proofs
\item Better definition for multisets, and an explanation of the bijection between multisets and histograms
\item Clarification sentence on the stability relation
\item Approach with floating point
\end{itemize}

\section{Mathematical operators}
\questionc{We would like to discuss how to approach this section further. We are not clear on the relation between MPFR and functions like \texttt{saturating\_add}.}

The notation we are using for various mathematical operations is, as of 24 June 2021, inspired by section 2 of \url{https://hal-ens-lyon.archives-ouvertes.fr/ensl-01529804/file/crlibm.pdf}. (The notation may change in the future.) We plan to use a similar standardized notation for describing the semantics of \href{https://www.mpfr.org/}{MPFR}.\footnote{Library used in OpenDP to deal with floating point arithmetic.}

\begin{definition}[$+, -, \times$]
    The symbols $+$, $-$, and $\times$ represent the usual mathematical operations of addition, subtraction, and multiplication, respectively.
\end{definition}

\begin{definition}[$\oplus, \ominus, \otimes$]
    The symbols $\oplus$, $\ominus$, and $\otimes$ denote the corresponding floating point operations in \href{https://www.mpfr.org/}{MPFR} / in Rust arithmetic. 
\end{definition}

\connor{Maybe change ``floating point operations in \href{https://www.mpfr.org/}{MPFR}'' to some other thing that it denotes. I'm not sure if they should ``denote the corresponding floating point operations in \href{https://www.mpfr.org/}{MPFR}'', but I wanted to preserve the notation, and I wasn't sure what it should be defined on \ldots.}
    
\silvia{Use same notation in both pseudocode and proofs?}

\todo{Change to cast.}\connor{I don't understand this todo; could someone else complete it?}
\begin{definition}
    Given a Rust value $x$, the function \texttt{val(x)} returns the actual (real or integer) number represented by the Rust value $x$.

\end{definition}

\horizline

\subsection{Notes, todos, questions}

\todo{Add the cast definition?}

\todo{max, min, val(x) (give number represented by Rust value), see email from June 21. Subindex for type?}

\todo{Add special float symbol for max, min?} No, from discussion on June 28: as long as we pass as arguments things that have used float symbols, it's OK.

\questionc{Type in subindex vs casting type.}
\connor{What does ``subindex for type'' mean (see todo above for use of this phrase; meaning possibly relates to question in this line)?} \silvia{I think it means deciding between, for example, $\texttt{max}_{\texttt{u32}}(x, y)$ or $\texttt{cast(max(x, y), u32)}$.}

\section{Data Representation}

\begin{definition}[Vector]
A vector $v$ is an ordered list of objects.
\end{definition}

\begin{definition}[Set]
A set is an unordered list of objects.
\end{definition}

\begin{definition}[Multiset]
A multiset is a modification of the notion of a set which, unlike a set, allows for repetitions for each of its elements. The number of repetitions of an element in the multiset corresponds to the \textit{multiplicity} of that element. \questionc{multiset vs MultiSet.} \silvia{Add what we said on July 15 (Connor).}
\end{definition}

%\silvia{Sets are by definition unordered, no need to specify that in the 2nd sentence.}

\begin{remark}
Given a vector $v$, we denote its multiset representation by $\MultiSet(v)$. This distinction is relevant because vectors are ordered objects whereas multisets are not. In the OpenDP library, all datasets are represented as vector domains, and therefore for any vector $v$ we need to use the notation $\MultiSet(v)$ when referring to its multiset representation to indicate that ordering should be dropped. \questionc{Clashes a bit with the $\MultiSet(\X)$ use in the PF; domain vs vector.} \silvia{This should be resolved after consensus on July 15.}
\end{remark}

\begin{example}
    Given $\MultiSet(2, 3, 3, 5, 5, 5 )$, element 2 has multiplicity 1, element 3 has multiplicity 2, and element 5 has multiplicity 3.
\end{example}

\questionc{Is this clear enough? Because this is very important.} \connor{Note: As of July 12, 2021, it may be less important since we may not need multisets anymore -- for example, we now have a ``vector-first'' definition for symmetric distance that we can use.} \silvia{Still needs discussion: defining terms in a ``vector-first" manner to avoid using $\MultiSet(v)$ in the proofs.}

\questionc{We should decide on whether we want to keep a vector approach or a multiset approach:}
\begin{definition}[Histogram notation, multiset version]
\label{defn:histogram}
    Let $h_x: \mathcal{X} \rightarrow \mathbb{N}$ be the histogram of a multiset $x \in \MultiSet(\mathcal{X} )$ for some domain $\mathcal{X}$. That is, $h_x(z)$ denotes the number of occurrences of $z \in \mathcal{X}$ in multiset $x$ (with multiplicities).
\end{definition}

%\todo{Change to vector domain notation. For vector $v$, $h_v(z)$ denotes the number of occurrences of $z$ in $\MultiSet(v)$.} \todo{Specify types: z?}

\begin{definition}[Histogram notation, vector version]
    For any vector $v$ of elements of a domain \texttt{D}, $h_v$ denotes the histogram of $v$. That is, for every element $z$ of type \texttt{T}, $h_v(z)$ denotes the number of occurrences of $z \in \texttt{D}$ in the entries of vector $v$ (with multiplicities).
\end{definition}

\todo{Say there is a bijection between multisets and histograms.}

%Given a vector $v$ in some vector domain, $h_v(z)$ denotes the number of occurrences of $z$ in $\MultiSet(v)$.

\section{Transformations \& Stability relation}
%\silvia{Transformations here? Section for them?}
\begin{definition}[Transformation]
    A transformation $T$ is a \textit{deterministic} mapping from arbitrary data types of derived data values to arbitrary data types of derived data values. In Rust, a transformation is specified by the following attributes: input domain, output domain, function, input metric, output metric, and stability relation. 
\end{definition}

See \href{https://www.overleaf.com/project/60d215bf90b337ac02200a99}{``List of definitions used in the pseudocode"} for further details of the pseudocode specification of a transformation.

%\silvia{Not c in R}
\begin{definition}[Stability parameter]\label{def:c}
    For some value of $c$, we say that a transformation $T$ is $c$\textit{-stable} if for all $x, x'$ in the input domain $\X$, and for input metric $d_{\X}$ and output metric $d_{\Y}$,
    \begin{equation}
        d_{\mathcal{Y}}(T(x), T(x')) \leq c \cdot d_{\mathcal{X}}(x, x').
    \end{equation}
    We say that $c$ is the \textit{stability parameter} of $T$.
\end{definition}

\silvia{Add a sentence saying that not all stability relations can be expressed only with $c$ (linear relationship).}

In the Rust code, the stability parameter $c$ (which in the case of clamping is equal to~1) gets wrapped up inside of the stability relation property, and the end user can test it empirically. \todo{Explain better -- but perhaps in the general guidelines doc?}

\begin{definition}[Stability relation]
    The \textit{stability relation}, denoted $\Relation(\din, \dout)$, is a boolean function which takes as input some $\din, \dout$ appropriately quantified and returns \texttt{True} if and only if the relation $\dout \geq c \cdot \din$ for some specified value of $c$. 
\end{definition}

We can also write the stability relation in the following form:

\begin{equation}
    \Relation(\din, \dout) = 
    \begin{cases} 
      \True & \textrm{if } \dout \geq c \cdot \din \\
      \False & \textrm{otherwise},
   \end{cases}
\end{equation}
specifying the concrete metrics $d_{\X}, d_{\Y}$ and types of $\din, \dout$ inside the function.

The associated type of $d_{\X}$ is equal to the type of $\din$, and the associated type of $d_{\Y}$ is equal to the type of $\dout$. Importantly, such relations are sound but \textit{not necessarily complete}. A transformation is considered \textit{valid} if its stability relation is sound.

%\begin{definition}[Relation]
%Relation($d_{in}$, $d_{out}$) means that \grace{Silvia can you add what you had for Relation here?}
%\end{definition}

\subsection{Stability for Randomness}


The following lemma and corollary are used to prove the stability guarantee holds in random transformations like \texttt{make\_impute}. 
\begin{lemma}[Stability for Randomness]
We define a randomized function $f: \texttt{DI} \rightarrow \texttt{DO}$. $\Relation(d_{in}, d_{out}) = True$ implies that for all $x, x' \in$ \texttt{DI} that are $d_{in}$-close, there exists a coupling $(R, R')$ of the randomness of $f(x)$ and $f(x')$ such that for all $(r, r') \in Support(R, R'),$ $f_r(x)$ is $d_{out}$-close to $f_{r'}(x')$.
\end{lemma}

\silvia{Proof? (Grace)}

The following corollary allows us to fix the random seed in random transformations to prove the stability guarantee.

\begin{corollary}
For randomized function $f: \texttt{DI} \rightarrow \texttt{DO}$, $\Relation(d_{in}, d_{out}) = True$ implies that for all $x, x' \in$ \texttt{DI} that are $d_{in}$-close and for all fixing of the randomness $r$ of $f$ (fixing seed of PRG), we have that $f_r(x)$ and $f_r(x')$ are $d_{out}$-close.
\end{corollary}

\horizline

\subsection{Notes, todos, questions}

\todo{Add explanation on forward and backward map?}

%\silvia{TODO: define first the customary definition for sym dist}
%\silvia{TODO: and make it correct for multisets}

\section{Metrics}
\silvia{Very important to always specify the types and domains when presenting metrics! (remarks from 24/6 Prof. Vadhan's OH) So:}

%\silvia{With that, changed multisets to vectors, but have we made it clear enough that this is how OpenDP is representing multisets? E.g., $u \Delta v$ is the multiset vs $u \Delta v$ is the vector corresponding to... (see below)}

\silvia{From meeting on July 13: each metric definition must contain its associated type and the list of possible domains it works with. This is not explicitly specified in Rust (i.e., the metric is only a name), but every time we find a transformation / measurement which uses the metric, include the corresponding domain in the definition if it is not there already. It should also include what it means to be $\din$ close under that metric.}

\connor{Because the Rust type on which a metric operates is covered in the pseudocode definitions, I think we just need to talk mathematically here, like ``symmetric distance is defined on vectors''.}

\silvia{On July 15 we settled on only writing the mathematical definition in this document, and then writing the Rust list of domains and associated types in the pseudocode definitions document.}

The associated type of any input metric is the type of the corresponding $\din$. In turn, the associated type of any output metric is the type of the corresponding $\dout$.

\subsection{Symmetric distance}
\begin{definition}[Symmetric difference]
The \textit{symmetric difference} between any two vectors $u, v$, denoted by $\MultiSet(u)\Delta \MultiSet(v)$, corresponds to the multiset representation of elements which are in either $u$ or $v$ but not in their intersection. The multiplicity of each element $x$ in $\MultiSet(u)\Delta \MultiSet(v)$ corresponds to the difference in absolute value of the multiplicities of $x$ in $\MultiSet(u)$ and in $\MultiSet(v)$.

%\connor{I think we should have an explanation of what is means for an element to be in, for example, $u$ but not in $v$. For example, if we have $u = \MultiSet(0,0,1)$ and $v = \MultiSet(0)$, is $0$ in $v$ or not? We want it to be that ``the first $0$ is in $v$, but the second $0$ is not in $v$'', but I don't think the current definition gets that message across.}

%\todo{Improve this definition}
\end{definition}

%\todo{Changed my mind after preconditions: better to say symmetric distance than \texttt{SymmetricDistance} (no longer appears in pseudocode)}

%\todo{Still unclear which information should be part of the metric. Mike said on 7/6 not to include the list of possible domains, but then where do we say $u, v$ belong to? Mike also said that he is not sure whether the \texttt{u32} should be part of the definition.}

%\silvia{Should change to \texttt{SymmetricDifference}, right?}

\silvia{In all these metric definitions, be careful about associated type and not to say input/output type. We never think of metrics as functions in the usual sense; instead, the associated type of distance is the type of $\din$ or $\dout$. When defining the metrics, we need to specify to which type of domains it applies.}


%\silvia{Cardinality of MultiSet(v). Comment on Slack about no input vs output type.}

\begin{example}
Because a multiset can have repeated elements, for $a = \MultiSet(1,2,1)$ and $b = \MultiSet(1,3)$, we have $a\Delta b = \MultiSet(1,3, 2)$. 
\end{example}

%\grace{I think there's a typo. Originally it said that $a\Delta b = \MultiSet(1,3,2)$.}

%\connor{Fixed now -- thanks for the catch!}

We introduce the notion of symmetric distance, which differs from symmetric difference in that it is the \emph{cardinality} of the multiset instead of the multiset itself.

\begin{definition}[Symmetric distance]
The \textit{symmetric distance} between any two vectors $u, v$, denoted $d_{Sym} = |\MultiSet(u) \Delta \MultiSet(v)|$, is equal to the cardinality of the symmetric difference between $u$ and $v$.
\end{definition}

%\connor{Do you think we should leave out the associated type above? The associated type may be more of a job for the pseudocode definitions since types aren't really mentioned in this doc.}

\silvia{The meaning of $d$-close under $d_{Sym}$ has been moved to the pseudocode definitions document.}

%\begin{definition}[$d$-close under $d_{Sym}$]
    %For any two vectors $u, v \in \texttt{VectorDomain(D)}$ and $d$ of type \texttt{u32}, we say that $u, v$ are $\din$-close under $d_{Sym}$ whenever $d_{Sym}(u, v) = |\MultiSet(u) \Delta \MultiSet(v)| \leq d$.
%\end{definition}

Note: symmetric distance is the metric which is used in the OpenDP library, while the definition symmetric difference is only included for completeness in this document.

% still working on this paragraph
Because there is a bijection between histograms and multisets, we can also define the \emph{symmetric distance} between $u$ and $v$ as the $\ell_1$ distance between the histograms for $u$ and $v$, denoted $h_u$ and $h_v$ (see Definition \ref{defn:histogram}). Then, we equivalently obtain that the symmetric distance between $u$ and $v$ is
$$d_{\text{Sym}}(u,v) = \lVert h_{v} - h_{w}\rVert_1 = \sum_{z\in \mathcal{X}} |h_u(z) - h_{v}(z)|.$$

The second equality follows from the definition of $\ell_1$ distance.

\silvia{Add the above as claim?}

\connor{Since these definitions are all written by us, I think we could say that we're defining symmetric distance as both things.}

\medskip
\textbf{Alternative definition of symmetric distance.}

Let $u,v$ be vectors of elements drawn from domain $\mathcal{X}$. Define $m_v(\ell)$ as the multiplicity of element $\ell$ in vector $v$. For example, if $v$ contains five instances of the number ``21'', then $m_v(21) = 5$.

An alternative definition of symmetric distance, then, is $$d_{\text{Sym}}(u,v) = \sum_{z\in \mathcal{X}} |m_u(z) - m_v(z)|.$$

\questionc{Which definition should we go with? Original definition, alternative definition, or both?}

\begin{claim}
Symmetric distance is a metric.
\end{claim}

Note that null data values are still counted in the symmetric distance. Adding or removing null values still influences the count.

\subsection{The path property of symmetric distance on sized domains}

\questionc{Given that we are avoiding texttt notation in this document, should we change \texttt{SizedDomain} for a mathematical definition of sized domain in the three lemmas that follow?}

\begin{lemma}[Path property of $d_{Sym}$ on \texttt{SizedDomain}]
    For any two vectors $v, w$ of the same length (that is, for any two vectors $v, w \in$ \texttt{SizedDomain(D)} for some domain \texttt{D}), $d_{Sym}(v,w)$ is an even integer; i.e., $d_{Sym}(v,w) = 2k$ for some integer $k \geq 0$. Moreover, there exist $k$ vectors $v=v_0, v_1, \ldots, v_k=w$ such that $d_{Sym}(v_i,v_{i+1})=2$ for all $i$.
\end{lemma}

\begin{lemma}
    For any two vectors $v, w \in$ \texttt{SizedDomain(D)}, $d_{Sym} = 2$ if and only if we can change one element of $v$ to obtain $v'$ such that $\MultiSet(v') = \MultiSet(w)$.
\end{lemma}

The above lemma follows directly from the definition of symmetric distance. 

\begin{lemma}
Given a function $f$ on input domain \texttt{SizedDomain(D)}, if $\dout(f(v), f(w)) \leq c$ for all vectors $v, w$ such that $d_{Sym}(v, w) = 2$, then $f$ is $c/2$-stable.
\end{lemma}

By the definition of $c$-stable (Definition~\ref{def:c}), 
that $f$ is $c/2$-stable is equivalent to stating that for all pairs $v', w'$ in the input domain, the following holds:
\[
    d_{\Y}(f(v'), f(w')) \leq c/2 \cdot d_{Sym}(v', w').
\]
Also equivalently, this means that the stability relation of $f$ is
\[
    \dout \geq c/2 \cdot \din.
\]

\todo{Add the more general version of path property by using shortest path distance on graphs as discussed on July 19. The proofs also need to be added.}

\subsection{The path property: a generalization}

\subsection{Substitute distance}
\begin{definition}[Substitute distance]
We only define the \emph{substitute distance} $d_{Sym}$ on multisets with the same number of elements. On two multisets $u,v\in \MultiSet(\mathcal{X})$ for some domain $\mathcal{X}$ where $|u| = |v|$, we say that the substitute distance $d_{Subs}(u,v)$ is equal to the cardinality of the relative complement $u \backslash v$, so $d_{Subs}(u,v) = |u\backslash v|$. (This can be thought of as fixing multiset $u$ and finding how many elements in $v$ are not represented in $v$.) \silvia{List of domains? Associated types? It is not in the library, so we cannot answer this -- maybe we should drop it.}

\connor{Substitute distance should have the same domain as symmetric distance since it is defined as half the symmetric distance. Probably the same ``associated type'', too, but I don't think we need to cover that here.}

Alternatively, we can define $d_{Subs}$ as


$$d_{Subs}(u,v) = \frac{1}{2}d_{\text{Sym}}(u,v).$$

Note that this metric is like a generalization of Hamming distance to multisets (recall that Hamming distance is defined on ordered objects). \silvia{If we say ``recall" then Hamming distance should be defined and explained.} \todo{Add Hamming? It is never used, so I would drop it.}
\end{definition}
\silvia{I would not define $d_{Subs}$ in terms of $d_{Sym}$; and instead write original definition and then add the 2-factor equality as a claim.}
\connor{I think they had defined $d_{subs}$ in terms of $d_{sym}$.}

\begin{claim}
Substitute distance is a metric.
\end{claim}

%\begin{claim}
%(2 factor of Subs and Sym)
%\end{claim}

\begin{remark}
    As of June 24, substitute distance has been removed from the library as a possible dataset metric. Therefore, symmetric distance is currently the only available dataset metric in the OpenDP library.
\end{remark}

\subsection{Absolute distance}

\begin{definition}[Absolute distance]\label{def:abs}
    Given two numbers $n, m$, the absolute distance between $n$ and $m$, denoted $d_{Abs}$, is defined as $d_{Abs}(n, m)= |n-m|$, where the horizontal bars represent absolute value.
\end{definition}

\begin{definition}[$\din$-close under $d_{Abs}$]
    For any two elements $n, m$ in \texttt{AllDomain(T)}, where \texttt{T} denotes an arbitrary type with trait \texttt{Sub(Output=T)}, and $\din$ of generic type \texttt{Q}, we say that $n, m$ are $\din$-close under $d_{Abs}$ whenever $d_{Abs}(n, m) = |n-m| \leq \din$.
\end{definition}

The same definition holds for $\dout$.

\begin{claim}
    Absolute distance is a metric.
\end{claim}

\subsection{Distances}
\begin{definition}[$k$-close]
For any metric $d$, we say that two elements $u, v$ are $k$-close under $d$ if $d(u, v) \leq k$.
\end{definition}

 \silvia{Important: $\din, \dout$ is more general than metrics; e.g., $(\epsilon, \delta)$-DP.}

For example, in the case of symmetric distance, vectors $u$ and $v$ are $k$-close whenever $d_{Sym}(u, v) = |\textrm{MultiSets}(v) \Delta \textrm{MultiSets}(u)| \leq k$. We remark that the type of $k$ must correspond to the associated type of $d$.

We remark that the notion of $k$-closeness can be defined more generally without relying on metrics and instead only using $\din, \dout$. If that is the case, it will be defined accordingly.

\horizline

\subsection{Notes, todos, questions}

\todo{Make sure to cite Programming Framework when required.}


\end{document}
